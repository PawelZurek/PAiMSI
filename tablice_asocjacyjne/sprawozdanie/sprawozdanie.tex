\documentclass[10 pt]{article}
\usepackage{graphicx}
\pagestyle{plain}
\usepackage[OT4]{polski}
\usepackage[utf8]{inputenc}
\title{Sprawozdzanie 4 \\ \emph{\textbf{Tablice Asocjacyjne}}}
\author{Paweł Żurek 200404}
\date{01.04.2014}
\begin{document}
\tableofcontents
\maketitle
\section{Wstęp}
Prosty program, w którym użyte są podstawowe funkcje tablic asocjacyjnych.
\section{Krótki opis programu}
\textbf{Program udostępnia prosty interfejs z następującymi opcjami: }
\begin{itemize}
\item Dodanie elementu wraz z kluczem
\item Usunięcie elementu o danym kluczu
\item Wyświetlenie elementu z danym kluczem
\item Sprawdzenia czy są jakieś elementy
\item Sprawdzenia dokładnej ilości elementów
\item Wyświetlenie elementów wraz z ich kluczami
\end{itemize}

\textbf{Najważniejsze elementy klasy głównej to : }
\begin{itemize}
\item Tablica typu string z wartościami elementów
\item Druga tablica, również typu string, która posiada klucze elementów
\end{itemize}

\section{Wnioski:}
\begin{itemize}
\item Program działa poprawnie, lecz wyszukiwanie klucza ma złożoność O(n), gdzie n to liczba elementów w tablicy
\item Jako budulec struktury użyłem tablicy opisanej za pomocą stosu, ponieważ z poprzednich zadań zauważyłem, iż ta struktura oprócz łatwej obsługi, działa dostatecznie szybko.

\end{itemize}


\end{document}