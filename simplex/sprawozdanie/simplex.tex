\documentclass[10 pt]{article}
\usepackage{graphicx}
\pagestyle{plain}
\usepackage[OT4]{polski}
\usepackage[utf8]{inputenc}
\title{Sprawozdzanie 9 \\ \emph{\textbf{Simplex v0.5}}}
\author{Paweł Żurek 200404}
\date{21.05.2014}
\begin{document}
\tableofcontents
\maketitle
\section{Wstęp}
Uproszczona wersja algorytmu Simplex, która pozwala znaleźć największą wartość zysku dla dwuwymiarowych funkcji celu.
\section{Krótki opis programu}
\textbf{Program prosi o : }
\begin{itemize}
\item wpisanie ilości równań
\item wpisanie poszczególnych współczynnikow
\item wpisanie poszczególnych ograniczen
\item wpisanie poszczególnych kosztów
\end{itemize}
Następnie wykonuje algorytm i na końcu wyświetla wynik.

\subsection{Uwagi :}
\begin{itemize}
\item Program działa tylko dla problemów DWU wymiarowych !!!!
\end{itemize}
\section{Algorytm }
Dzialanie funkcji:
\begin{itemize}
\item obliczenie kluczowych punktów
\item obliczenie wartości zysku w tych punktach
\item wybor punktu, w którym zysk jest najlepszy
\end{itemize}

\section{Wnioski:}
\begin{itemize}
\item Program działa dobrze, lecz tylko dla przypadków dwu wymiarowych
\item Kod jest skomplikowany, przez swoją prostotę. Tzn, jest zrobiony z prostych elementów, lecz ułożonych w skopmlikowany sposób
\end{itemize}

Dokładna dokumentacja jest dostępna w pliku \textit{dokumentacja.pdf} ( wygenerowana w \LaTeX 'u.

\end{document}